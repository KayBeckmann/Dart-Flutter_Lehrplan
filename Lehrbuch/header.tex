% Unicode-Unterstützung (lualatex)
\usepackage{fontspec}
\setmainfont{Latin Modern Roman}
\setsansfont{Latin Modern Sans}
\setmonofont{DejaVu Sans Mono}[Scale=0.85]

% Farben
\usepackage{xcolor}
\definecolor{linkblue}{HTML}{2563EB}
\definecolor{codegreen}{HTML}{16A34A}
\definecolor{codegray}{HTML}{6B7280}
\definecolor{codebg}{HTML}{F3F4F6}
\definecolor{darkblue}{HTML}{1E3A5F}
\definecolor{accentblue}{HTML}{3B82F6}

% Hyperlinks (klickbares Inhaltsverzeichnis) - hyperref wird von pandoc geladen
\AtBeginDocument{%
  \hypersetup{
    colorlinks=true,
    linkcolor=darkblue,
    urlcolor=linkblue,
    bookmarks=true,
    bookmarksnumbered=true,
    bookmarksopen=true,
    bookmarksopenlevel=2,
    pdfstartview=FitH,
    pdftitle={Dart & Flutter -- Lehrbuch},
    pdfauthor={Lehrplan},
    pdfsubject={Frontend & Backend Entwicklung mit Dart und Flutter}
  }%
}

% Kopf- und Fusszeilen
\usepackage{fancyhdr}
\pagestyle{fancy}
\fancyhf{}
\fancyhead[LE,RO]{\thepage}
\fancyhead[RE]{\leftmark}
\fancyhead[LO]{\rightmark}
\renewcommand{\headrulewidth}{0.4pt}
\fancypagestyle{plain}{
  \fancyhf{}
  \fancyfoot[C]{\thepage}
  \renewcommand{\headrulewidth}{0pt}
}

% Bessere Tabellen
\usepackage{booktabs}
\usepackage{longtable}

% Inhaltsverzeichnis-Tiefe
\setcounter{tocdepth}{2}
\setcounter{secnumdepth}{3}

% Code-Blöcke: Zeilenumbruch bei langen Zeilen
\usepackage{fvextra}
\fvset{breaklines=true, breakanywhere=true, breaksymbolright=\small\ensuremath{\hookrightarrow}}
\usepackage{framed}
\definecolor{shadecolor}{HTML}{F8F9FA}

% Deckblatt
\usepackage{titling}

% Kapitel-Styling
\usepackage{titlesec}
\titleformat{\chapter}[display]
  {\normalfont\huge\bfseries\color{darkblue}}
  {\chaptertitlename\ \thechapter}{20pt}{\Huge}
\titlespacing*{\chapter}{0pt}{-20pt}{30pt}

% Part-Styling
\titleformat{\part}[display]
  {\centering\normalfont\Huge\bfseries\color{darkblue}}
  {\partname\ \thepart}{20pt}{\huge}

% Bessere Listen
\providecommand{\tightlist}{%
  \setlength{\itemsep}{0pt}\setlength{\parskip}{0pt}}
